\documentclass{article}
\usepackage[utf8]{inputenc}
\usepackage{geometry}
\usepackage{polski}
\usepackage{float}
\usepackage{graphicx}
\usepackage[shortlabels]{enumitem}
\usepackage{amsmath}
\usepackage{amsfonts}

\usepackage{hyperref}

\title{Raport 1}
\author{Aleksander Jakóbczyk i Bogdan Banasiak\\ 
	Nr indeksu: 255939 i !!!!!!!!!!!!!!!!}
\date{}\date{}

\newtheorem{theorem}{Theorem}
\newtheorem{definition}{Definition}

\DeclareMathOperator{\sign}{sign}

\begin{document}
	
	\maketitle
	\section*{title}
	\section*{Information and formulas}
		\subsection*{Stable random variable}
		There are two parameterizations of a random variable from an alpha stable distribution $S(\alpha, \beta , \gamma, \delta; 0)$ and $S(\alpha, \beta , \gamma, \delta; 1)$.
		They are uniquely determined by the characteristic function.
		\begin{definition} A random variable $X$ is stable if and only if $X=^d aZ +b$, with \\$\alpha \in (0,2],\;\beta\in [-1,1],\; a\ne1,\; b\in\mathbb{R}$ and $Z$ is a random variable with characteristic function  
			\begin{gather}
				\varphi_Z(u) = \exp(i u Z) =
				\begin{cases}\label{def:case Z}
					\exp\left(- |u|^\alpha(1-i\beta\tan(\frac{\pi\alpha}{2})(\sign u)  \right) &\alpha \ne 1,\\
					\exp\left(- |u|^\alpha(1+i\beta\frac{2}{\pi}(\sign u)\ln|u|  \right) &\alpha = 1.
				\end{cases}
			\end{gather}
		\end{definition}
		\begin{definition} Let $X \sim S(\alpha, \beta , \gamma, \delta; 0)$ with $\alpha \in (0,2]$, $\beta \in [-1,1]$, $\gamma \ge 0$, $\delta\in\mathbb{R}$ then  
			\begin{gather*}
				X =^d 
				\begin{cases}
					\gamma (Z- \beta\tan(\frac{\pi\alpha}{2})+\delta)& \alpha\ne1,\\
					\gamma Z + \delta& \alpha=1,
				\end{cases}
			\end{gather*}
			where $Z = Z(\alpha,\beta)$ is given by \ref{def:case Z}.
		\end{definition}

		\begin{definition} Let $X \sim S(\alpha, \beta , \gamma, \delta; 1)$ with $\alpha \in (0,2]$, $\beta \in [-1,1]$, $\gamma \ge 0$, $\delta\in\mathbb{R}$ then  
			\begin{gather*}
				X =^d 
				\begin{cases}
					\gamma Z + \delta & \alpha \ne 1,\\
					\gamma Z + (\delta + \beta\frac{2}{\pi}\ln\gamma)& \alpha = 1,
				\end{cases}
			\end{gather*}
			where $Z = Z(\alpha,\beta)$ is given by \ref{def:case Z}.
		\end{definition}

		Above we defined the general stable law in the 0-parameterization and 1-parameterization.
		Alternatively, we can swap between the parameterizations using the following theorem;
		\begin{theorem}Let $Z\sim S(\alpha,\beta,1,0;0)$  with $\alpha \in (0,2]$, $\beta \in [-1,1]$, $\gamma \ge 0$, $\delta\in\mathbb{R}$ then  
			\begin{gather*}
				\begin{cases}
					\gamma Z + \delta + \beta \gamma \tan\left(\frac{\pi\alpha}{2}\right) &\alpha\ne1,\\
					\gamma Z + \delta + \beta \frac{2}{\pi}\ln\gamma  &\alpha=1
				\end{cases}\sim S(\alpha,\beta,\gamma,\delta; 1),
			\end{gather*}
		\end{theorem}

		The tail exponent estimation method gives us the information
		about the index of stability. The tails of stable random variable are asymptotically power laws. 
		\begin{theorem}[Tail approximation] Let $X \sim S(\alpha, \beta , \gamma, \delta; k)$ with $\alpha \in (0,2)$, $\beta \in (-1,1]$, $k=0,1$ then as~$x\to \infty$:
			\begin{align*}
				1 - F_X(x) &\sim \gamma^\alpha c_a (1+\beta)x^{-\alpha},\\
				f_X(x) &\sim \alpha \gamma^\alpha c_a (1+\beta) x^{-(\alpha + 1)}
			\end{align*}
			where $c_a = \sin(\frac{\pi\alpha}{2})\Gamma(\alpha)/\pi$ and $f(x)\sim g(x)$ as $x\to a$ means $\lim_{x\to a} h(x)/f(x) = 1$. Using the reflection property, the lower tail properties are
			similar: for $\beta\in[-1,1)]$ as $x \to \infty$:
			\begin{align*}
				F_X(-x) &\sim  \gamma^\alpha c_a (1-\beta)x^{-\alpha},\\
				f_X(-x) &\sim  \alpha \gamma^\alpha (1-\beta)c_a x^{-(\alpha + 1)}
			\end{align*}
		\end{theorem}
	
\end{document}